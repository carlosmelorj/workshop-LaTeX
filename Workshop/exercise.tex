\documentclass[a4paper]{article}
% If the aim is to make another type of document, just change where it says {article}

\usepackage[english]{babel}
\usepackage[utf8]{inputenc}
\usepackage{amsmath}
\usepackage{graphicx}
\usepackage{hyperref}
\usepackage{listings}
% Here you can add all the packages that exist in the world. To avoid problems of overlap or complexity, it is best to use only those needed for the document. 

\usepackage{color}

\definecolor{codegreen}{rgb}{0,0.6,0}
\definecolor{codegray}{rgb}{0.5,0.5,0.5}
\definecolor{backcolour}{rgb}{0.95,0.95,0.92}

\lstdefinestyle{codestyle}{
	basicstyle=\ttfamily,
    keywordstyle=\color{blue}\ttfamily,
    stringstyle=\color{red}\ttfamily,
    commentstyle=\color{magenta}\ttfamily,
    morecomment=[l][\color{codegreen}]{\#},
    backgroundcolor=\color{backcolour},
    breakatwhitespace=false,
    breaklines=true,
    captionpos=b,
    keepspaces=true,
    numbers=left,
    numbersep=5pt,
    showspaces=false,
    showstringspaces=false,
    showtabs=false,
    tabsize=2
}

% Don't touch anything so far. These are basic settings. :)

\title{IEEE UP SB Code Week Workshop}

\author{}%add your name

\date{\today} %\today you have the date always updated. You can change it with text when you want it to have a specific date.


\begin{document}
\maketitle

%\begin{abstract}
%In the abstract, one should explain very briefly the content of the paper.
%\end{abstract}

\section{Introduction}

Your introduction goes here! Some examples of commonly used commands and features are listed below, to help you get started.

\section{Some \LaTeX{} Examples}
\label{sec:examples}

\subsection{How to Include Figures}

Use the includegraphics command to include images. Use the figure environment and the caption command to add a number and a caption to your figure.

So if I remember correctly, figure  is \LaTeX{}.

\subsection{How to Make Tables}

Use the table and tabular commands for basic tables --- see Table , for example.


\subsection{How to Make Sections and Subsections}

Use section and subsection commands to organize your document. \LaTeX{} handles all the formatting and numbering automatically. Use ref and label commands for cross-references.

\subsection{How to Make Lists}

You can make lists with automatic numbering \dots


\subsection{How to Write Mathematics}

\LaTeX{} is great at typesetting mathematics. Let $X_1, X_2, \ldots, X_n$ be a sequence of independent and identically distributed random variables with
$\text{E}[X_i] = \mu$ and $\text{Var}[X_i] = \sigma^2 < \infty$, and let denote their mean. Then as $n$ approaches infinity, the random variables $\sqrt{n}(S_n - \mu)$ converge in distribution to a normal $\mathcal{N}(0, \sigma^2)$.

\subsection{How to add Citations and a References List}

You can upload a \verb|.bib| file containing your BibTeX entries, created with JabRef; or import your \href{https://www.overleaf.com/blog/184}{Mendeley}, CiteULike or Zotero library as a \verb|.bib| file. You can then cite entries from it, like this: . Just remember to specify a bibliography style, as well as the filename of the \verb|.bib|.

You can find a \href{https://www.overleaf.com/help/97-how-to-include-a-bibliography-using-bibtex}{video tutorial here} to learn more about BibTeX.

We hope you find Overleaf useful, and please let us know if you have any feedback using the help menu above --- or use the contact form at \url{https://www.overleaf.com/contact}!


Everybody loves conclusions.


%\bibliographystyle{alpha}
%\bibliography{sample}

\end{document}
